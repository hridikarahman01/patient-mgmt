\documentclass[]{article}
\usepackage{lmodern}
\usepackage{amssymb,amsmath}
\usepackage{ifxetex,ifluatex}
\usepackage{fixltx2e} % provides \textsubscript
\ifnum 0\ifxetex 1\fi\ifluatex 1\fi=0 % if pdftex
  \usepackage[T1]{fontenc}
  \usepackage[utf8]{inputenc}
\else % if luatex or xelatex
  \ifxetex
    \usepackage{mathspec}
  \else
    \usepackage{fontspec}
  \fi
  \defaultfontfeatures{Ligatures=TeX,Scale=MatchLowercase}
\fi
% use upquote if available, for straight quotes in verbatim environments
\IfFileExists{upquote.sty}{\usepackage{upquote}}{}
% use microtype if available
\IfFileExists{microtype.sty}{%
\usepackage{microtype}
\UseMicrotypeSet[protrusion]{basicmath} % disable protrusion for tt fonts
}{}
\usepackage[unicode=true]{hyperref}
\hypersetup{
            pdfborder={0 0 0},
            breaklinks=true}
\urlstyle{same}  % don't use monospace font for urls
\usepackage{graphicx,grffile}
\makeatletter
\def\maxwidth{\ifdim\Gin@nat@width>\linewidth\linewidth\else\Gin@nat@width\fi}
\def\maxheight{\ifdim\Gin@nat@height>\textheight\textheight\else\Gin@nat@height\fi}
\makeatother
% Scale images if necessary, so that they will not overflow the page
% margins by default, and it is still possible to overwrite the defaults
% using explicit options in \includegraphics[width, height, ...]{}
\setkeys{Gin}{width=\maxwidth,height=\maxheight,keepaspectratio}
\IfFileExists{parskip.sty}{%
\usepackage{parskip}
}{% else
\setlength{\parindent}{0pt}
\setlength{\parskip}{6pt plus 2pt minus 1pt}
}
\setlength{\emergencystretch}{3em}  % prevent overfull lines
\providecommand{\tightlist}{%
  \setlength{\itemsep}{0pt}\setlength{\parskip}{0pt}}
\setcounter{secnumdepth}{0}
% Redefines (sub)paragraphs to behave more like sections
\ifx\paragraph\undefined\else
\let\oldparagraph\paragraph
\renewcommand{\paragraph}[1]{\oldparagraph{#1}\mbox{}}
\fi
\ifx\subparagraph\undefined\else
\let\oldsubparagraph\subparagraph
\renewcommand{\subparagraph}[1]{\oldsubparagraph{#1}\mbox{}}
\fi

\date{}

\begin{document}

\section{\texorpdfstring{\textbf{\emph{Project
Outline}}}{Project Outline}}\label{project-outline}

Design an opportunity for universal access to healthcare from anywhere
at anytime

\begin{itemize}
\item
  \includegraphics[width=2.60000in,height=2.60000in]{media/image1.png}Patient
  Management System

  \begin{itemize}
  \item
    Patient Portal
  \item
    Doctor Database
  \item
    Patient Feedback
  \item
    Common Disease and Efficient medicine tracker

    \begin{itemize}
    \item
      Online Platform
    \end{itemize}
  \end{itemize}
\end{itemize}

\section{\texorpdfstring{\textbf{\emph{Current
Scenario}}}{Current Scenario}}\label{current-scenario}

Bangladesh is South Asian densely populated low middle-income economy.
Like many others in her peer group, healthcare in Bangladesh struggles
in a vicious cycle of inefficiency, mismanagement, patient
dissatisfaction, and high costs. Reluctantness to embrace digital
technologies in the healthcare sector, staff shortages, and not
involving the consumers (patients) in the decision-making process lead
to this stage.

\section{\texorpdfstring{\textbf{\emph{Solution}}}{Solution}}\label{solution}

This HealthTech project automates daily chores in a hospital that is
advantageous over time-consuming hand-operated practices. Through this,
doctors get access to the patients' archives stored in the database.
This web application empowers the patients to book an appointment, call
an ambulance, and even give a view on the satisfaction of the services
received. Generating invoices and recording information about the
diagnostics given to the patient can also be done.

\section{\texorpdfstring{\textbf{\emph{Objective}}}{Objective}}\label{objective}

\begin{enumerate}
\def\labelenumi{\arabic{enumi}.}
\item
  To build a patient-centered healthcare system by giving them a voice,
  accessibility of healthcare from anywhere at any time.
\item
  To keep tabs on common diseases and efficacious treatments.
\item
  Empower physicians to analyze patients beforehand from their stored
  medical records.
\end{enumerate}

\section{\texorpdfstring{\textbf{\emph{Target
Customers}}}{Target Customers}}\label{target-customers}

\begin{itemize}
\item
  Patients: People who will use our services to book an appointment or
  call an ambulance. They can get to know all the information of the
  doctor and service providers. They can also provide feedback on their
  visit to the healthcare center.
\item
  Doctors and Medical Center owners: The owners can keep a tab of all
  the bookings. They will be able to know which diseases are on rising.
  Furthermore, they can learn about the quality of their provided
  services. The doctors can study the patient's medical history
  beforehand and build a triage system in a time of emergency.
\end{itemize}

\section{\texorpdfstring{\textbf{\emph{Value
Proposition}}}{Value Proposition}}\label{value-proposition}

The project presents a compilation of interactive features for a patient
looking to book an appointment, call an ambulance in case of an
emergency. It will lessen the bothering and time of going to the
hospital to book an appointment. In developing nations like Bangladesh,
few institutions have sought patients' views on satisfaction with
services, and there is little effort to involve them in measuring
satisfaction or defining health service standards. Our project aims to
give them a voice by allowing them to evaluate the services received.
Our country has a shortage of hospital staff, yet every doctor uses
trained staff to assist them in patient booking and handling their
diagnostic reports. As they do everything through pen and paper, it
results in improper workforce utilization, mismanagement, and longer
waiting times for patients. However, here, the bookings can be confirmed
immediately by one person. The patients' medical history data will
assist the doctor to develop accuracy, become motivated, gain beforehand
understanding, and build confidence.

\section{\texorpdfstring{\textbf{\emph{Web Application Features and
Description}}}{Web Application Features and Description}}\label{web-application-features-and-description}

The web page will open with a Login or Sign up page view. New patients
have to register to avail the services. When patients register or logs
in, they can:

\begin{itemize}
\item
  Choose a doctor and desired time slot if available.
\item
  Book an appointment for diagnostic tests.
\item
  Call an ambulance.
\item
  Access history of their past visits to a doctor.
\item
  After visiting a doctor, they can rate or give suggestions.
\end{itemize}

Hospital administrators or doctors can also log in from the initial
page. After logging in, they can:

\begin{itemize}
\item
  See the patient bookings and confirm them.
\item
  See the previously recorded details of the patient.
\item
  Store the patient's prescription.
\item
  See the rank of most common diseases and administered drugs.
\item
  Generate bills.
\item
  Feedback on their services.
\end{itemize}

\section{\texorpdfstring{\textbf{\emph{Tools and Resources
}}}{Tools and Resources }}\label{tools-and-resources}

HTML, CSS, JavaScript, MySQL, PHP, Web servers, APIs.

\section{\texorpdfstring{\textbf{\emph{Challenges}}}{Challenges}}\label{challenges}

The whole process is online, and a large portion of the conservative
demographic hesitates to adopt new technology. That's why we must use
simple and interactive UI. High expenses of modern servers, maintenance
costs, providing security for online payment, training the staff can be
challenging. Data verification will also be a problem. Bangladesh has
started digital NID cards, but accessing the database will not be easy.
We may need to follow a lengthy and tedious procedure. Lastly, gaining
the trust of the patient may also be challenging.

\end{document}
